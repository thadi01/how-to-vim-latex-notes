% more character support
\usepackage[utf8]{inputenc}
\usepackage[T1]{fontenc}
\usepackage[english]{babel}
% adds support for a lot of symbols
\usepackage{textcomp, gensymb, amssymb, latexsym}
% ever so slightly improves visual of text
\usepackage{microtype}
% to handle images
\usepackage{graphicx}
% path for images (used by graphicx package)
\graphicspath{ {./images/} }
% math packages
\usepackage{amsmath, amsfonts, mathtools, amsthm}
% lets you bold mathematical symbols well
\usepackage{bm}
% adds colour support
\usepackage[dvipsnames]{xcolor}
% adds a diagonal line through something
\usepackage{cancel}
% add hyper links
\usepackage{hyperref}
% frame around envs
\usepackage{thmtools, tcolorbox}
\usepackage[framemethod=TikZ]{mdframed}
% makes things fancy!
\usepackage{fancyhdr}
% for editing the page
\usepackage[margin=1in, hmargin=1in]{geometry}
% last page
\usepackage{lastpage}

% short cuts for certain symbols like complex or real numbers
\newcommand{\N}{\ensuremath{\mathbb{N}}}
\newcommand{\R}{\ensuremath{\mathbb{R}}}
\newcommand{\Z}{\ensuremath{\mathbb{Z}}}
\renewcommand{\O}{\ensuremath{\emptyset}}
\newcommand{\F}{\ensuremath{\mathbb{F}}}
\newcommand{\C}{\ensuremath{\mathbb{C}}}
% quickly right trig fun + theta
\newcommand{\Sin}{\sin,\theta}
\newcommand{\Cos}{\cos,\theta}
\newcommand{\Tan}{\tan,\theta}
%Make implies and impliedby shorter
\let\implies\Rightarrow
\let\impliedby\Leftarrow
\let\iff\Leftrightarrow
\let\epsilon\varepsilon 


% styling pages
\pagestyle{fancy} 
\fancyhf{}
\fancyhead[L]{\nouppercase{\leftmark}}  
\fancyhead[R]{\nouppercase{\rightmark}} 
\fancyfoot[C]{\thepage\ of \protect\pageref*{LastPage}}
\renewcommand{\footrulewidth}{0.4pt}

% framing environments
\mdfsetup{skipabove=1em,skipbelow=1em}

\declaretheoremstyle[
    headfont=\bfseries\sffamily\color{BurntOrange!75!black}, notefont=\large, bodyfont=\normalfont,
    mdframed={
        linewidth=2pt,
        rightline=false, topline=false, bottomline=false,
        linecolor=Dandelion, backgroundcolor=Goldenrod!5,
    },
    postheadspace=\newline
]{thmyellowbox}

\declaretheoremstyle[
    headfont=\bfseries\sffamily\color{Mulberry!70!black}, notefont=\large, bodyfont=\normalfont,
    mdframed={
        linewidth=2pt,
        rightline=false, topline=false, bottomline=false,
        linecolor=Mulberry, backgroundcolor=Mulberry!5,
    },
    postheadspace=\newline
]{thmpurplebox}

\declaretheoremstyle[
    headfont=\bfseries\sffamily\color{NavyBlue!70!black}, notefont=\large, bodyfont=\normalfont,
    mdframed={
        linewidth=2pt,
        rightline=false, topline=false, bottomline=false,
        linecolor=NavyBlue
    },
    postheadspace=\newline
]{thmbluebox}

\declaretheoremstyle[
    headfont=\bfseries\sffamily\color{ForestGreen!70!black}, notefont=\large, bodyfont=\normalfont,
    mdframed={
    linewidth=2pt,
    rightline=false, topline=false, bottomline=false,
    linecolor=ForestGreen, backgroundcolor=ForestGreen!5,
    },
    postheadspace=\newline
]{thmgreenbox}

% framing for different environment
\declaretheorem[name=Definition, parent=section, style=thmyellowbox]{defn}

\declaretheorem[name=Theorem, sibling=defn, style=thmpurplebox]{thm}

\declaretheorem[name=Lemma, sibling=defn, style=thmbluebox]{lem}

\declaretheorem[name=Corollary, sibling=defn, style=thmbluebox]{cor}

\declaretheorem[name=Example, sibling=defn, style=thmgreenbox]{eg}
