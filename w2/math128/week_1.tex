\chapter{Supply and Demand}

\section{Demand}

\begin{definition}
  \textbf{Demand} is the quantity of something we want, and the ability to procure.
\end{definition}

What impacts our demand?
\begin{enumerate}
  \item Price\begin{itemize}
    \item Higher price $\rightarrow$ lower demand
  \end{itemize}
  \item Utility and preference\begin{itemize}
    \item Higher utility $\rightarrow$ higher demand
  \end{itemize}
  \item Price of alternatives\begin{itemize}
    \item Higher demand of complementary goods $\rightarrow$ higher demand
    \item Higher demand of substitute goods $\rightarrow$ lower demand
  \end{itemize}
  \item Income\begin{itemize}
    \item Higher income $\rightarrow$ higher demand
  \end{itemize}
  \item Expected future price\begin{itemize}
    \item Higher expected future price $\leftrightarrow$ higher demand
    \item Price doesn't actually need to change for demand to be affected
  \end{itemize}
  \item Information\begin{itemize}
    \item What people know affects demand
    \item Creates self-fulfilling prophecies
  \end{itemize}
  \item Government interventions\begin{itemize}
    \item Restrictions $\rightarrow$ lower demand
  \end{itemize}
\end{enumerate}

\begin{theorem}
  (The ``law'' of demand) If prices go up, demand goes down, and vice versa.

  Luxury goods violate this.
\end{theorem}

How can we represent demand?
\begin{enumerate}
  \item Graph\begin{itemize}
    \item Price on $y$-axis
    \item Quantity on $x$-axis
    \item Generally downwards sloping
    \item Best for presentation
  \end{itemize}
  \item Function\begin{itemize}
    \item A function that maps independent variables (price, income, etc.) to quantity demanded
    \item More precise, good for prediction
    \item Can use more than one factor at once
  \end{itemize}
\end{enumerate}
\[D(p)=(-)p\]

\subsection{Supply}

What impacts supply?
\begin{enumerate}
  \item Costs of inputs\begin{itemize}
    \item Higher costs $\rightarrow$ lower Supply
  \end{itemize}
  \item Price\begin{itemize}
    \item Higher prices $\rightarrow$ higher supply
  \end{itemize}
  \item Government interventions
\end{enumerate}

How can we represent supply? \[S(p,c)=(+)p+(-)c\]

\subsection{Summing supply \& demand}

\begin{itemize}
  \item Add horizontally
\end{itemize}

When can we use the supply \& demand model?
\begin{enumerate}
  \item Minimal to no frictions
  \item Full information
  \item Everyone is a price taker
  \item Identical products
\end{enumerate}

\subsection{Solving the supply \& demand model}

Find the quantity and price where $Q_D=Q_S$, or quantity supplied equals quantity demanded.

For graphs, find where supply \& demand interesects.

For functions, set $Q_D=Q_S$ and solve for $p$.

\subsection{Competitive model}

\begin{definition}[Market structure]
  \begin{itemize}
    \item Number of firms in the market
    \item Ease with which firms can enter/exit
    \item Ease with which firms can differentiate their products
  \end{itemize}
\end{definition}

\begin{definition}[Perfectly competitive market]
  A market in perfect competition has the following properties:

  \begin{itemize}
    \item Numerous small firms
    \item Easy entry and exit
    \item Difficult to differentiate
    \item Everyone is a price taker
    \item Everyone is profit maximizing (a legal responsibility)
    \item Firms are competitive in the short run and in the long run
  \end{itemize}
\end{definition}

In a perfectly competitive market, the market has normal supply and demand curves. For firms, the demand is perfectly elastic at the market price (residual demand) because they are price takers.

\subsection{Profit}
\[\text{Profit} = R(q)-C(q) = \pi\]

Economic profit also considers opportunity cost, the value of the best foregone alternative.

Firm profit vs quantity graphs look like a parabola. There is an optimal quantity that maximizes firm profit.

Options to maximize profit:

\begin{enumerate}
  \item Choose $q$ that maximizes profit
  \item Find $q$ for which marginal profit is $0$
  \item Find $q$ for which marginal cost is equal to price
\end{enumerate}

\begin{definition}
  The term \textbf{marginal} represents the value of a $1$ unit increase.
\end{definition}

\subsection{Costs}

\section{Taxes, Tariffs, Surplus}

\subsection{Surplus}

\begin{definition}
  \textbf{Surplus} is the value gained from buying/selling something for less/more than its worth to you.
\end{definition}

Supplier surplus refers to the area above $S$ but below $p$.

Consumer surplus refers to the area below $D$ but above $p$.

There are two types of tax:
\begin{itemize}
  \item Per unit (fixed tax rate)
  \item Ad Valorem (percent tax rate)
\end{itemize}
